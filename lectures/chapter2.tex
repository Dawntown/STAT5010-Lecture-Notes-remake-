\lesson{original Lecture 3}{10 Ways of Viewing a Random Variable}
\subsection{Familiar Perspectives}

\begin{enumerate}
    \item probability space
    \item random variable
    \item distribution
    \item $\cdots$
\end{enumerate}


\begin{definition}[Probability space]
    $(\Omega,\mathcal{F},P)$, 
    where $\Omega$ is sample space with element $\omega$,
    $\Omega$ is $\sigma$-algebra/$\sigma$-field, and 
    $P$ is probability measure.
\end{definition}
\info{\scriptsize refer to Measure Theory (Probability and Measures, Billingsley, 1995)}

\subsection{Characteristic Functions (ch.f.)}

\begin{definition}[ch.f.]
    \begin{gather}
        \Phi_X(t)=\mathbb{E}(e^{itX})=\int{e^{itx}}dF(x)
    \end{gather}
    where $i=\sqrt{-1}$ and $e^{itx}=\cos{tx}+i\sin{tx}$.
\end{definition}

\begin{example}
    The characteristic function of $Z\sim{\mathcal{N}(0,1)}$:
    \begin{gather}
        \Phi_Z(t)=\exp{\{-\frac{1}{2}t^2\}}
    \end{gather}
\end{example}

\begin{align}
    |\mathbb{E}e^{itX}| 
    \leq& \mathbb{E}|e^{itX}| \\
    =& \mathbb{E}|\cos{(tX)}+i\sin{(tX)}| \\
    =& \mathbb{E}\sqrt{\cos^2(tX)+\sin^2(tX)}=1
\end{align}

Inversion formula:
\begin{align}
    f_X(x)=&\frac{1}{2\pi}\int_{-\infty}^\infty{e^{-itx}\Phi_X(t)}dt \\
    F_X(x) - F_X(y)=&\frac{1}{2\pi}\int_{-\infty}^\infty\frac{e^{-ity}-e^{-itx}}{it}\Phi_X(t)dt
\end{align}
for the continuous points of $F_X(x)$ and $F_X(y)$.

\begin{theorem}
    \textbf{Uniqueness of ch.f.}\\
    Let $X$ and $Y$ be random $k$-vectors, then
    \begin{enumerate}
        \item If $Phi_X(t)=\Phi_Y(t),~\forall{t}\in\mathbb{R}^k$, then $F_X=F_Y$;
        \item If $M_X(t)=M_Y(t)<\infty,~\forall{t}~s.t.~|t|<h$, then $F_X=F_Y$
    \end{enumerate}
\end{theorem}
\begin{proof}
$~$\\
    \begin{enumerate}
        \item \label{prf:uniqchf1} For any $\boldsymbol{a}=(a_1, \cdots, a_k)^T\in\mathbb{R}^k$, $\boldsymbol{b}\in\mathbb{R}^k$,
        and $(\boldsymbol{a},\boldsymbol{b}]=(a_1,b_1]\times\cdots\times(a_k,b_k]$
        satisfying $P_X(\text{the boundary of}~(\boldsymbol{a},\boldsymbol{b}])=0$,
        \begin{gather}
            P((\boldsymbol{a},\boldsymbol{b}])
            =\lim_{c\to\infty}\int_{-c}^c\cdots\int_{-c}^c{
            \frac{\Phi_X(t_1,\cdots,t_k)}{(-1)^{\frac{k}{2}(2\pi)^k}}
            }\prod_{j=1}^k\frac{e^{-it_ja_j}-e^{-it_jb_j}}{it_j}dt_j
        \end{gather}
        \item \label{prf:uniqchf2} First, consider the case where $k=1$. Since
        \begin{gather}
            e^{s|x|}\leq e^{sx}+e^{-sx},
        \end{gather}
        we conclude that $|X|$ has an MGF that is finite in the neighborhood $(-c,c)$ for some $c>0$,
        and that $|X|$ has finite moments of all orders.\\
        Observe that 
        \begin{gather}
            \left| e^{itx}\left[ e^{iax}-\sum_{j=0}^n\frac{(iax)^j}{j!} \right] \right| \leq \frac{|ax|^{n+1}}{(n+1)!},
        \end{gather}
        We can write
        \begin{gather}
            \left| \Phi_X(t+a)-\sum_{j=0}^n\frac{a^j}{j!}\mathbb{E}[(iX)^je^{itX}] \right| \leq \frac{|a|^{n+1}\mathbb{E}|X|^{n+1}}{(n+1)!}
        \end{gather}
        
        By the fact (\ref{eq:fact2}) below, we have
        \begin{align}
            \mathbb{E}(e^{i(t+a)X})
            =& \mathbb{E}\left( e^{itX}\sum_{j=0}^\infty\frac{a^j}{j!}(iX)^j \right)\\
            =& \sum_{j=0}^\infty\frac{a^j}{j!}\mathbb{E}\left( e^{itX}(iX)^j \right)\\
            =& \sum_{j=0}^\infty\frac{a^j}{j!}\Phi^{(j)}_X(t)
        \end{align}
        
        The result/expansion also holds for $Y$. Under the assumption that $M_X=M_Y<\infty$ in a neighbourhood of 0,
        $X$ and $Y$ have the same moments of all orders. 
        By the Fact 2., 
        $\Phi^{(j)}_X(0)=\Phi_Y^{(j)}(0)$ for all $j=1,2,\cdots$, 
        which and Equation (\ref{eq:fact2}) with $t=0$ imply that $\Phi_X$ and $\Phi_Y$ are the same on the interval $(-c,c)$.
        
        Considering $t=c-\epsilon$, and $-c+\varepsilon$ for all arbitrary small $\varepsilon>$ in Equation (\ref{eq:fact2})
        shows that $\Phi_X$ and $\Phi_Y$ also agree on $(-2c+\varepsilon, 2c-\varepsilon)$ and hence $(-2c,2c)$.
        Likewise, the same argument $\Phi_X$ and $\Phi_Y$ are the same on $(-3c,3c)$ and so on.
        Hence, $\Phi_X(t)=\Phi_Y(t)$ for all $t$ and by Proof part (\ref{prf:uniqchf1}) $F_X=F_Y$.
        
        For the case $k\geq 2$, if $F_X\neq F_Y$, then by Proof part (\ref{prf:uniqchf1}),
        there exists $t\in\mathbb{R}^k$ such that $\Phi_X(t)\neq\Phi_Y(t)$, Then $\Phi_{t^TX}(1)\neq\Phi_{t^TY}(1)$,
        which implies that $F_{t^TX}=F_{t^TY}$.
        
        But $M_X=M_Y<\infty$ in a neighbourhood of $\boldsymbol{0}\in\mathbb{R}^k$ implies that $M_{t^TX}=M_{t^TY}<\infty$
        in a neighbourhood of $0\in\mathbb{R}$ and by the proved result for $k=1$, $M_{t^TX}=M_{t^TY}$. The contradiction show that $F_X=F_Y$.
    \end{enumerate}
\end{proof}
$~$\\
\textbf{Facts}:
\begin{enumerate}
    \item If $M_X(t)$ is finite in a neighbourhood of 0, 
    then $\mathbb{E}(X_1^{r_1} \cdots X_k^{r_k})$ is finite for any integers $r_1,\cdots,r_k\geq 0$ and 
    $M_X(t)$ has the power series expansion
    \begin{gather}
        M_X(t)=\sum_{(r_1,\cdots,r_k)\in\mathbb{Z}^+\cup{\{0\}}}
        \frac{\mathbb{E}(X_1^{r_1} \cdots X_k^{r_k})t_1^{r_1} \cdots t_k^{r_k}}{r_1!\cdots r_k!}
    \end{gather}
    for any $k$-dimension random vector $X$.
    \item \begin{gather}
        \frac{\partial^r\Phi_X(t)}{\partial{t_1^{r_1}}\cdots\partial{t_k^{r_k}}}
        =(-1)^\frac{r}{2}\mathbb{E}(X_1^{r_1} \cdots X_k^{r_k}e^{it^TX})
    \end{gather}
    with 
    \begin{gather}
        \left.\frac{\partial^r\Phi_X(t)}{\partial{t_1^{r_1}}\cdots\partial{t_k^{r_k}}}\right|_{t=0}
        =(-1)^\frac{r}{2}\mathbb{E}(X_1^{r_1} \cdots X_k^{r_k})
    \end{gather}
    we can write, for any $t\in\mathbb{R}$, 
    \begin{gather}
        \Phi_X(t+a)=\sum_{j=0}^\infty\frac{\Phi_X^{(j)}(t)}{j!}a^j, ~~|a|<c\label{eq:fact2}
    \end{gather}
\end{enumerate}

\subsection{Conditional Probability}
\begin{gather}
    P(B|A)=\frac{P(A\cup{B})}{P(A)},~\text{when}~P(B)\neq 0
\end{gather}
where by convention, $P(B|A)=0$ when $P(A)=0$.
Suppose $\mathbb{E}|f(X,Y)|<\infty$,
\begin{gather}
    \mathbb{E}f(X,Y)=\mathbb{E}[\mathbb{E}f(X,Y)|X]\label{eq:towerexpec}
\end{gather}\info{\scriptsize Tower expectation: $\mathbb{E}(Y|X)=\beta_0^TX$}
with 
\begin{gather}
    \mathbb{E}[f(X,Y)|X]\triangleq h(X)~\text{and}\\
    H(x)=\mathbb{E}[f(X,Y)|X=x]=\int{f(x,y)}dQ_x(y)
\end{gather}
where $Q_x(y)$ denotes the conditional distribution of $Y$ given $X=x$. 
Then Equation (\ref{eq:towerexpec}) becomes
\begin{gather}
    \mathbb{E}f(X,Y)=\int{H(x)}dP_X(x)=\int\int{f(x,y)}dQ_x(y)dP_X(x)
\end{gather}
where $P_X(x)$ is the distribution of $X$.

\begin{definition}[NFFC]
    The function $Q$ is a conditional distribution of $Y$ given $X$, writte as $Y|X=x\sim Q_x$ if
    \begin{enumerate}[{(1)}]
        \item $Q_x(\cdot)$ is a probability measure for all $x$,
        \item $Q_x(B)$ is a measurable function of $x$ for nay Borel set $B$,
        \item for any Borel sets $A$ and $B$,
        \begin{gather}
            P(X\in A, Y\in B)=\int_A{Q_x(B)}dP_X(x)
        \end{gather}
    \end{enumerate}
\end{definition}


\subsection{Tail behaviour}
For a scalar random variable $X$ with pdf $f$, 
we say $X$ has
\begin{enumerate}
    \item an exponential tail if 
    \begin{gather}
        \lim_{a\to\infty}-\frac{\log{(1-F(a))}}{ca^r}=1~~\text{for some}~c>0,r>0
    \end{gather}
    \item an algebraic tail if 
    \begin{gather}
        \lim_{a\to\infty}-\frac{\log{(1-F(a))}}{m\log{a}}=1~~\text{for some}~m>0.
    \end{gather}
\end{enumerate}

\begin{example}
    $~$
    \begin{itemize}
        \item Exponential: $F(a)=1-e^{-\lambda a}$, $c=\lambda,r=1$
        \item Gaussian: $F(a)=\cdots$, $c=2,r=2$
        \item Student t: heavy-tail density for small d.f.
    \end{itemize}
\end{example}
\improvement{\scriptsize Not finished but can be ignored}

\newpage