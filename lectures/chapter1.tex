\lesson{original Lecture 1 -- 3}{Overview of Statistical Inference}

Different disciplines require to convert data to 
inference/conclusion via statistics:
\begin{itemize}
    \item genetic, medical, earth sciences
    \item economics, finance
    \item networks, images, recordings
\end{itemize}
$~$\\
% \\ \hspace*{\fill} \\
Questions concerned:
\begin{itemize}
    \item capture the uncertainty. {\color{gray}signal+noises}
    \item methodology
    \begin{itemize}
        \item point estimator 
        {\color{gray}e.g. 
        $\hat{\beta}=
        (\boldsymbol{X}^T\boldsymbol{X})^{-1}\boldsymbol{X}^T\boldsymbol{y}$, where $\boldsymbol{y}$ is r.v..
        }
        \item testing $\sim$ confidence region
        {\color{gray} e.g.
        $H_0:~\theta=\theta_0$ vs $H_1:~\theta=\theta_1$
        }
    \end{itemize}
    \item optimality / optimal inference
    \begin{itemize}
        \item finite-sample optimality 
        {\color{gray}
        iid obs: $x_1,\cdots,x_n$
        }
        \item asymptotic properties 
        {\color{gray} 
        $n\to\infty$
        }
    \end{itemize}
\end{itemize}

\subsection{Decision Theory }%\citep{Wald1939}}

\begin{gather}
    (X_1,\cdots,X_n)\overset{iid}{\sim}F~(\text{or}~f)
\end{gather}
A statistical model is a family of distributions $\mathbb{P}$ indexed by
a parameter $\theta$. We demote
\begin{gather}
    \mathbb{P}=\{p_\theta:~\theta\in\Omega\}
\end{gather}
where $\Omega\in\mathbb{R}^k$ is parameter space.

\begin{example}
    Flipping a coin.\\
    
    obs (data): a sequence of coin flips $(X_1,\cdots,X_n)$. 
    One can write $\mathbb{P}=\{\text{Bernoulli}(\theta): \theta\in[0,1]\triangleq\Omega\}$ and $P_\theta(X_i=1)=\theta$\\
    
    data $\boldsymbol{X}$\\
    $\Rightarrow$ estimator: $\hat{\theta}(\boldsymbol{X})$ for the parameter $h(\theta)$ 
    {\color{gray} e.g. ($\bar{X}=\frac{1}{n}\sum_{i=1}^nX_i$)}\\
    $\Rightarrow$ $\hat{\theta}(\boldsymbol{X})$ for estimation and testing.
\end{example}

\subsubsection{A Decision Procedure}
\begin{definition}[Estimator]
    $\delta:\mathcal{X}\to\mathcal{D}$, 
    a map from sample space $\mathcal{X}$ to decision space $\mathcal{D}$.
\end{definition}

\begin{example}
    Take $\mathbb{P}=\{\text{Bernoulli}(\theta)\}$ as shown earlier,
    \begin{enumerate}
        \item \textbf{Estimating $\theta$:} 
        the dicision space is $\mathcal{D}=[0,1]$ and 
        one possible decision procedure can be 
        $\delta(\boldsymbol{X})=\bar{X}_n$.
        \item \textbf{Hypothesis testing:} 
        Accept/Reject the Null hypothesis $H_0:\theta=\frac{1}{2}$. 
        Correspondingly, the decision space is $\mathcal{D}=\{\text{Accept, Reject}\}$.
        One possible procedure is $\delta(\boldsymbol{X})=$
        \textit{``Reject to if $\bar{X}_n > 0.5$ and accept it otherwise''}.
        \item A loss function is mapping $L: \Omega\times\mathcal{D} \to \mathbb{R}^+\cup{\{0\}}$.
        $L(\theta, d)$  represents the penalty for making the decision $d$ 
        when $\theta$ is in fact the true parameter for the distribution generating the data.
    \end{enumerate}
\end{example}

Loss function
\begin{itemize}
    \item Squared error loss: $L(\theta,d)=(\theta-d)^2$
    \item Abstract error loss: $L(\theta,d)=|\theta-d|$
    \item Risk function: 
    $L(\theta,d)=\int{\omega(\theta)(\theta-d)^2d\Lambda(\theta)}$
\end{itemize}

Frequentist perspective: $\theta$ is fixed yet unknown;
Bayesian perspective: $\Theta\sim$ distribution.

\begin{definition}[Risk function]
    Overall weighted average of $L(\theta,\delta)$
    \begin{gather}
        R(\theta, \delta)=\mathbb{E}_\theta L(\theta, \delta(\boldsymbol{X}))
    \end{gather}
\end{definition}

\begin{definition}[Admissibility]
\label{def:admissible}
    $\delta$ is inadmissible if there exists $\delta'$ such that 
    \begin{enumerate}[{(1)}]
        \item $R(\theta, \delta') \leq R(\theta, \delta)~\forall{\theta}\in\Omega$ and 
        \item $R(\theta', \delta') < R(\theta', \delta)~\text{for some}~\theta'\in\Omega$
    \end{enumerate}
    Otherwise, the $\delta$ is admissible.
\end{definition}
\info{\scriptsize
You should be superior to any other applicant in at least one aspect, 
or you will not be admitted.}

\subsubsection{Types of Model}
\begin{enumerate}
    \item Parametric models: Bernoulli($\theta$), $\mathcal{N}(\mu,\sigma^2)$, t($k$), ...
    \item Non-parametric models: empirical cdf $\hat{F}_n(x)=\frac{1}{n}\sum_{i=1}^n{I(X_i\leq{x})}$
    \item Semi-parametric models: Cox proportional hazard model
    % \begin{align}
    %     \lambda(t|Z(t))&=\lambda_0(t)\exp{(\beta^TZ(t))}
    %     &=\frac{f(t|Z)}{}
        
    % \end{align}
\end{enumerate}


\subsection{Large-Sample Theory}
\subsubsection{Convergence in probability}
\begin{definition}[Convergence in probability]
    A sequence of random variable $Y_n$ (e.g. $\delta_n(\boldsymbol{X})=\frac{1}{n}\sum_{i=1}^n{X_i}$) 
    converges in probability to a random variable $Y$ as $n\to\infty$,
    written as $Y_n\overset{p}{\longrightarrow}Y$,
    if $\forall{\varepsilon}>0$
    \begin{gather}
        P(|Y_n-Y|\geq \varepsilon)\overset{n\to\infty}{\longrightarrow}0
    \end{gather}
\end{definition}

\begin{example}
    \textbf{Weak Law of Large Numbers (WLLNs)}\\
    $X_1,\cdots,X_n\overset{iid}{\sim}(\mu,\sigma^2),~\sigma^2<\infty,\bar{X}_n=\frac{1}{n}\sum_{i=1}^n{X_i}$,
    $\mathbb{E}(\bar{X}_n-\mu)^2=\mathrm{Var}\bar{X}_n=\frac{\sigma^2}{n} \to 0$ as $n\to\infty$,
    in which case $\bar{X}_n\overset{p}{\longrightarrow}\mu$ as $n\to\infty$.
\end{example}

\begin{prop}
    If $f(\cdot)$ is continuous at $c$, and if $Y_n\overset{p}{\longrightarrow}c$, then $f(Y_n)\overset{p}{\longrightarrow}f(c)$.
\end{prop}
\begin{proof}
    Since $f$ is continuous at $c$, given any $\varepsilon > 0$, 
    then exists $\delta_\varepsilon > 0$ such that $|f(y)-f(c)|<\varepsilon$ 
    whenever $|y-c|<\delta_\varepsilon$. 
    Thus
    \begin{gather}
        P(|Y_n-c|<\delta_\varepsilon)\leq P(|f(y_n)-f(c)|<\varepsilon)
    \end{gather}
    which implies
    \begin{gather}
        P(|f(y_n)-f(c)|\geq\varepsilon) \leq P(|Y_n-c|>\delta_\varepsilon) \to 0
    \end{gather}
    as $Y_n\overset{p}{\longleftarrow}$ as $n\to\infty$.
\end{proof}
\info{\scriptsize $\delta-\varepsilon$ language}

\subsubsection{Convergence in distribution}
\begin{definition}[Convergence in distribution]
    A sequence of r.v.'s $Y_n~(n \geq 1)$ with cdf's $H_n$ converges
    in distribution (in law) to a random variable $Y$ with cdf $H$ if
    \begin{gather}
        H_n(y)\to H(y)
    \end{gather}
    as $n\to\infty$ whenever $H$ continuous at $y$. 
    For notation, we write $Y_n\Rightarrow Y$ or $Y_n\overset{d/l}{\longrightarrow}Y$.
\end{definition}

\begin{example}
    Suppose $Y_n=\frac{1}{n}$, a degenerate r.v. and that $Y$ is always 0.
    Then 
    \begin{gather}
        H_n(y)=P(Y_n\leq y)=\mathbb{I}(\frac{1}{n}\leq y).
    \end{gather}
    If $y>0$, then $H_n(y)=\mathbb{I}(\frac{1}{n}\leq y)\to 1$ as $n\to\infty$
    for eventually $\frac{1}{n}$ will be less than $y$.
    If $y\leq 0$, then $H_n(y)=\mathbb{I}(\frac{1}{n}\leq y)=0$ for all $n$ and 
    so $H_n(y)\to 0$ as $n\to\infty$.
    \newline
    Note that $H(y)=P(Y\leq y)=\mathbb{I}(0\leq y)$. Comparisons with
    the limits just obtained show that $H_n(y)\to H(y)$ if $y \neq 0$, 
    but $H_n(0) = 0 \to 0 \neq 1 = H(0)$. 
    So, in this example, $Y_n\overset{d}{\longrightarrow}Y$
    but the cumulative distribution function do not converge to $H(y)$ as $y=0$.
\end{example}

\newpage